\section*{|| Most interesting personal projects ||}
	\subsection*{\underline{Compiler from scratch:}\space \href{https://github.com/birth-software/bloat-buster}{\textbf{github.com/birth-software/bloat-buster}}}
	
	\paragraph{}Bloat Buster is the latest of several compilers I have developed in my spare time. Both the first and the last utilize LLVM, while the one in the middle was fully implemented from scratch. I have also experimented with C transpilation, primitive register allocation, and basic optimization passes, as well as attempting to vectorize an assembler. The lexer, parser, and semantic analysis were always written manually, without the use of generators. The latest is the most ambitious of them all, currently supporting only x86\_64-linux, and has successfully self-hosted with the help of LLVM.
	
	\subsection*{\underline{Kernel from scratch:}\space \href{https://github.com/birth-software/birth}{github.com/birth-software/birth}}
	Several very basic kernels written from scratch, from which Birth is the most advanced. It supports x86\_64 and aims to be an efficient, simple and secure multikernel. There are plans to port to other mainstream high-performance CPU architectures. Currently it's on hold waiting for a compiler to be ready as the operating system is planned to be rewritten in the language I am designing.